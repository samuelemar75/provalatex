\documentclass{beamer}
\usepackage[utf8]{inputenc}
\usepackage[T1]{fontenc}
\usepackage{lmodern}
\usepackage{listings}
\usepackage{hyperref}

% Impostazioni per la visualizzazione del codice
\lstset{
  basicstyle=\footnotesize\ttfamily,
  breaklines=true,
}

\title{Percorso Didattico PHP:\\ Dalle Basi all'Interazione con Database}
\author{}
\date{}

\begin{document}

\frame{\titlepage}

% Slide 1: Obiettivi
\begin{frame}
\frametitle{Obiettivi}
\begin{itemize}
    \item Introdurre PHP e l'ambiente di sviluppo
    \item Approfondire la programmazione con PHP (variabili, strutture di controllo, funzioni, array)
    \item Gestire form e interagire con database SQL
    \item Affrontare esempi pratici e progetti finali
\end{itemize}
\end{frame}

% Slide 2: Agenda
\begin{frame}
\frametitle{Agenda della Presentazione}
\begin{enumerate}
    \item Modulo 1: Introduzione a PHP e Ambiente di Sviluppo
    \item Modulo 2: Tipi di Dati, Operatori e Strutture di Controllo
    \item Modulo 3: Funzioni e Array
    \item Modulo 4: Gestione dei Form e Interazione Base
    \item Modulo 5: Interazione con Database SQL (MySQL)
    \item Modulo 6: Progetto Finale Integrato
    \item Modulo 7: Approfondimenti e Sicurezza
\end{enumerate}
\end{frame}

% ====================
% Modulo 1
% ====================

\begin{frame}
\frametitle{Modulo 1: Introduzione a PHP e Ambiente di Sviluppo}
\textbf{Concetti Base e Ambiente:}
\begin{itemize}
    \item Introduzione a PHP (linguaggio lato server)
    \item Installazione e configurazione di XAMPP/WAMP/MAMP
    \item Struttura minima di un file PHP: \texttt{\textless ?php ... ?\textgreater}
\end{itemize}
\end{frame}

\begin{frame}[fragile]
\frametitle{Modulo 1: Esempi Base}
\textbf{Hello World:}
\begin{lstlisting}
<?php
  echo "Hello, World!";
?>
\end{lstlisting}
\vspace{0.5em}
\textbf{Variabili e Concatenazione:}
\begin{lstlisting}
<?php
  $nome = "Mario";
  $eta = 16;
  echo "Il mio nome è " . $nome . " e ho " . $eta . " anni.";
?>
\end{lstlisting}
\end{frame}

\begin{frame}[fragile]
\frametitle{Modulo 1: Esempi Aggiuntivi}
\textbf{Commenti in PHP:}
\begin{lstlisting}
<?php
  // Questo è un commento su una riga
  /* Questo è un commento
     su più righe */
  echo "Commenti in PHP!";
?>
\end{lstlisting}
\vspace{0.5em}
\textbf{Inclusione di File Esterni:}
\begin{lstlisting}
// File: header.php
<?php
  echo "<h1>Benvenuti al corso PHP</h1>";
?>
\end{lstlisting}
\vspace{0.5em}
\textbf{Visualizzazione della Data e Ora:}
\begin{lstlisting}
<?php
  echo "Oggi è " . date("d/m/Y") . " e l'ora attuale è " . date("H:i:s");
?>
\end{lstlisting}
\end{frame}

% ====================
% Modulo 2
% ====================

\begin{frame}
\frametitle{Modulo 2: Tipi di Dati, Operatori e Strutture di Controllo}
\begin{itemize}
    \item Tipi di dati: stringhe, numeri, booleani
    \item Strutture condizionali: \texttt{if}, \texttt{else}, \texttt{switch}
    \item Cicli: \texttt{for}, \texttt{while}, \texttt{foreach}
\end{itemize}
\end{frame}

\begin{frame}[fragile]
\frametitle{Modulo 2: Esempi Base}
\textbf{Verifica Pari/Dispari:}
\begin{lstlisting}
<?php
  $numero = 7;
  if ($numero % 2 == 0) {
      echo "$numero è pari.";
  } else {
      echo "$numero è dispari.";
  }
?>
\end{lstlisting}
\vspace{0.5em}
\textbf{Ciclo For (1-10):}
\begin{lstlisting}
<?php
  for ($i = 1; $i <= 10; $i++) {
      echo $i . " ";
  }
?>
\end{lstlisting}
\vspace{0.5em}
\textbf{Calcolo del Fattoriale:}
\begin{lstlisting}
<?php
  $numero = 5;
  $fattoriale = 1;
  for ($i = 1; $i <= $numero; $i++) {
      $fattoriale *= $i;
  }
  echo "Il fattoriale di $numero è $fattoriale.";
?>
\end{lstlisting}
\end{frame}

\begin{frame}[fragile]
\frametitle{Modulo 2: Esempi Aggiuntivi}
\textbf{Calcolo della Media di un Array:}
\begin{lstlisting}
<?php
  $numeri = [10, 20, 30, 40, 50];
  $somma = array_sum($numeri);
  $media = $somma / count($numeri);
  echo "La media è: " . $media;
?>
\end{lstlisting}
\vspace{0.5em}
\textbf{Utilizzo del Costrutto switch:}
\begin{lstlisting}
<?php
  $giorno = 3;
  switch ($giorno) {
      case 1:
          echo "Lunedì";
          break;
      case 2:
          echo "Martedì";
          break;
      case 3:
          echo "Mercoledì";
          break;
      default:
          echo "Altro giorno";
  }
?>
\end{lstlisting}
\vspace{0.5em}
\textbf{Verifica se un Numero è Primo:}
\begin{lstlisting}
<?php
  $num = 29;
  $isPrimo = true;
  if ($num < 2) {
      $isPrimo = false;
  } else {
      for ($i = 2; $i <= sqrt($num); $i++) {
          if ($num % $i == 0) {
              $isPrimo = false;
              break;
          }
      }
  }
  echo $num . ($isPrimo ? " è primo." : " non è primo.");
?>
\end{lstlisting}
\end{frame}

% ====================
% Modulo 3
% ====================

\begin{frame}
\frametitle{Modulo 3: Funzioni e Array}
\begin{itemize}
    \item Definizione e utilizzo di funzioni in PHP
    \item Array indicizzati, associativi e multidimensionali
\end{itemize}
\end{frame}

\begin{frame}[fragile]
\frametitle{Modulo 3: Esempi Base}
\textbf{Funzione per Sommare Elementi di un Array:}
\begin{lstlisting}
<?php
  function sommaArray($arr) {
      $somma = 0;
      foreach ($arr as $valore) {
          $somma += $valore;
      }
      return $somma;
  }
  $numeri = [1, 2, 3, 4, 5];
  echo "La somma è: " . sommaArray($numeri);
?>
\end{lstlisting}
\vspace{0.5em}
\textbf{Array Associativo di Studenti:}
\begin{lstlisting}
<?php
  $studenti = [
      "Mario" => 16,
      "Lucia" => 17,
      "Giovanni" => 16
  ];
  foreach ($studenti as $nome => $eta) {
      echo "Studente: $nome, Età: $eta<br>";
  }
?>
\end{lstlisting}
\end{frame}

\begin{frame}[fragile]
\frametitle{Modulo 3: Esempi Aggiuntivi}
\textbf{Funzione per Trovare il Massimo in un Array:}
\begin{lstlisting}
<?php
  function massimoArray($arr) {
      $max = $arr[0];
      foreach ($arr as $valore) {
          if ($valore > $max) {
              $max = $valore;
          }
      }
      return $max;
  }
  $numeri = [3, 5, 7, 2, 9];
  echo "Il massimo è: " . massimoArray($numeri);
?>
\end{lstlisting}
\vspace{0.5em}
\textbf{Array Multidimensionale (Studenti con Voti):}
\begin{lstlisting}
<?php
  $studenti = [
      ["nome" => "Mario", "voto" => 8],
      ["nome" => "Lucia", "voto" => 9],
      ["nome" => "Giovanni", "voto" => 7]
  ];
  foreach ($studenti as $studente) {
      echo "Studente: " . $studente['nome'] . " - Voto: " . $studente['voto'] . "<br>";
  }
?>
\end{lstlisting}
\vspace{0.5em}
\textbf{Ricerca di un Valore in un Array:}
\begin{lstlisting}
<?php
  function trovaIndice($arr, $valoreRicercato) {
      foreach ($arr as $indice => $valore) {
          if ($valore == $valoreRicercato) {
              return $indice;
          }
      }
      return -1;
  }
  $numeri = [10, 20, 30, 40];
  echo "L'indice di 30 è: " . trovaIndice($numeri, 30);
?>
\end{lstlisting}
\end{frame}

% ====================
% Modulo 4
% ====================

\begin{frame}
\frametitle{Modulo 4: Gestione dei Form e Interazione Base}
\begin{itemize}
    \item Creazione di form HTML
    \item Differenza tra metodi GET e POST
    \item Validazione e sanitizzazione dei dati
\end{itemize}
\end{frame}

\begin{frame}[fragile]
\frametitle{Modulo 4: Esempi Base}
\textbf{Form di Contatto Semplice:}
\begin{itemize}
    \item \textbf{index.html}
    \begin{lstlisting}[language=HTML]
<!DOCTYPE html>
<html>
<head>
    <meta charset="utf-8">
    <title>Form di Contatto</title>
</head>
<body>
    <form action="processa.php" method="post">
        Nome: <input type="text" name="nome"><br>
        Email: <input type="email" name="email"><br>
        <input type="submit" value="Invia">
    </form>
</body>
</html>
    \end{lstlisting}
    \item \textbf{processa.php}
    \begin{lstlisting}
<?php
  $nome = isset($_POST['nome']) ? trim($_POST['nome']) : '';
  $email = isset($_POST['email']) ? trim($_POST['email']) : '';
  if (empty($nome) || empty($email)) {
      echo "Tutti i campi sono obbligatori.";
      exit;
  }
  if (!filter_var($email, FILTER_VALIDATE_EMAIL)) {
      echo "L'email inserita non è valida.";
      exit;
  }
  echo "Nome: $nome<br>Email: $email";
?>
    \end{lstlisting}
\end{itemize}
\end{frame}

\begin{frame}[fragile]
\frametitle{Modulo 4: Esempi Aggiuntivi}
\textbf{Form di Login Semplice:}
\begin{itemize}
    \item \textbf{login.html}
    \begin{lstlisting}[language=HTML]
<!DOCTYPE html>
<html>
<head>
    <meta charset="utf-8">
    <title>Login</title>
</head>
<body>
    <form action="verifica_login.php" method="post">
        Username: <input type="text" name="username"><br>
        Password: <input type="password" name="password"><br>
        <input type="submit" value="Accedi">
    </form>
</body>
</html>
    \end{lstlisting}
    \item \textbf{verifica_login.php}
    \begin{lstlisting}
<?php
  $username = isset($_POST['username']) ? htmlspecialchars(trim($_POST['username'])) : '';
  $password = isset($_POST['password']) ? htmlspecialchars(trim($_POST['password'])) : '';
  if ($username === "admin" && $password === "1234") {
      echo "Accesso effettuato!";
  } else {
      echo "Credenziali errate.";
  }
?>
    \end{lstlisting}
\end{itemize}
\vspace{0.5em}
\textbf{Form di Ricerca con Metodo GET:}
\begin{itemize}
    \item \textbf{ricerca.html}
    \begin{lstlisting}[language=HTML]
<!DOCTYPE html>
<html>
<head>
    <meta charset="utf-8">
    <title>Ricerca</title>
</head>
<body>
    <form action="risultati.php" method="get">
        Cerca: <input type="text" name="query"><br>
        <input type="submit" value="Cerca">
    </form>
</body>
</html>
    \end{lstlisting}
    \item \textbf{risultati.php}
    \begin{lstlisting}
<?php
  $query = isset($_GET['query']) ? htmlspecialchars(trim($_GET['query'])) : '';
  echo "Risultati per la ricerca: " . $query;
?>
    \end{lstlisting}
\end{itemize}
\vspace{0.5em}
\textbf{Form per l'Invio di Commenti:}
\begin{lstlisting}[language=HTML]
<!DOCTYPE html>
<html>
<head>
    <meta charset="utf-8">
    <title>Invia Commento</title>
</head>
<body>
    <form action="salva_commento.php" method="post">
        Nome: <input type="text" name="nome"><br>
        Commento: <textarea name="commento"></textarea><br>
        <input type="submit" value="Invia Commento">
    </form>
</body>
</html>
\end{lstlisting}
\end{frame}

% ====================
% Modulo 5
% ====================

\begin{frame}
\frametitle{Modulo 5: Interazione con Database SQL (MySQL)}
\begin{itemize}
    \item Creazione e gestione di database relazionali (es. MySQL)
    \item Utilizzo di PDO per connessione sicura
    \item Prevenzione della SQL injection con query preparate
\end{itemize}
\end{frame}

\begin{frame}[fragile]
\frametitle{Modulo 5: Esempio Base di Connessione e Visualizzazione}
\begin{lstlisting}
<?php
  $host = 'localhost';
  $dbname = 'scuola';
  $user = 'root';
  $password = '';
  try {
      $pdo = new PDO("mysql:host=$host;dbname=$dbname;charset=utf8", $user, $password);
      $pdo->setAttribute(PDO::ATTR_ERRMODE, PDO::ERRMODE_EXCEPTION);
      $stmt = $pdo->query("SELECT * FROM utenti");
      $utenti = $stmt->fetchAll(PDO::FETCH_ASSOC);
      foreach ($utenti as $utente) {
          echo "ID: " . $utente['id'] . " - Nome: " . $utente['nome'] . " - Email: " . $utente['email'] . "<br>";
      }
  } catch (PDOException $e) {
      echo "Connessione fallita: " . $e->getMessage();
  }
?>
\end{lstlisting}
\end{frame}

\begin{frame}[fragile]
\frametitle{Modulo 5: Esempi Aggiuntivi di Operazioni CRUD}
\textbf{Inserimento in una Tabella 'prodotti':}
\begin{lstlisting}
<?php
  try {
      $pdo = new PDO("mysql:host=localhost;dbname=negozio;charset=utf8", 'root', '');
      $pdo->setAttribute(PDO::ATTR_ERRMODE, PDO::ERRMODE_EXCEPTION);
      $stmt = $pdo->prepare("INSERT INTO prodotti (nome, prezzo) VALUES (:nome, :prezzo)");
      $stmt->execute([
          ':nome' => 'Prodotto A',
          ':prezzo' => 19.99
      ]);
      echo "Prodotto inserito!";
  } catch (PDOException $e) {
      echo "Errore: " . $e->getMessage();
  }
?>
\end{lstlisting}
\vspace{0.5em}
\textbf{Aggiornamento di un Record:}
\begin{lstlisting}
<?php
  try {
      $pdo = new PDO("mysql:host=localhost;dbname=negozio;charset=utf8", 'root', '');
      $pdo->setAttribute(PDO::ATTR_ERRMODE, PDO::ERRMODE_EXCEPTION);
      $stmt = $pdo->prepare("UPDATE prodotti SET prezzo = :prezzo WHERE id = :id");
      $stmt->execute([
          ':prezzo' => 24.99,
          ':id' => 1
      ]);
      echo "Prodotto aggiornato!";
  } catch (PDOException $e) {
      echo "Errore: " . $e->getMessage();
  }
?>
\end{lstlisting}
\vspace{0.5em}
\textbf{Eliminazione di un Record:}
\begin{lstlisting}
<?php
  try {
      $pdo = new PDO("mysql:host=localhost;dbname=negozio;charset=utf8", 'root', '');
      $pdo->setAttribute(PDO::ATTR_ERRMODE, PDO::ERRMODE_EXCEPTION);
      $stmt = $pdo->prepare("DELETE FROM prodotti WHERE id = :id");
      $stmt->execute([':id' => 1]);
      echo "Prodotto eliminato!";
  } catch (PDOException $e) {
      echo "Errore: " . $e->getMessage();
  }
?>
\end{lstlisting}
\end{frame}

% ====================
% Modulo 6
% ====================

\begin{frame}
\frametitle{Modulo 6: Progetto Finale Integrato}
\textbf{Proposta di Progetto: Rubrica/Contatti}
\begin{itemize}
    \item Inserimento, visualizzazione, modifica ed eliminazione di contatti
    \item Possibilità di ricerca per nome o altri campi
    \item Attività: Progettazione del database, interfaccia HTML/CSS, logica CRUD con PHP e MySQL
\end{itemize}
\end{frame}

\begin{frame}
\frametitle{Modulo 6: Altre Idee di Progetto}
\begin{enumerate}
    \item \textbf{Blog Semplice:} Gestione di post e commenti, operazioni CRUD.
    \item \textbf{To-Do List:} Creazione e gestione di attività, marcatura come completate, eliminazione.
    \item \textbf{Gestione Eventi:} Inserimento ed aggiornamento di eventi, registrazione partecipanti, filtri e ricerche.
\end{enumerate}
\end{frame}

% ====================
% Modulo 7
% ====================

\begin{frame}
\frametitle{Modulo 7: Approfondimenti e Sicurezza}
\begin{itemize}
    \item Gestione degli errori con \texttt{try-catch} e configurazione di \texttt{error\_reporting}
    \item Sanitizzazione e validazione avanzata degli input
    \item Prevenzione di vulnerabilità: SQL injection, XSS, CSRF
\end{itemize}
\end{frame}

\begin{frame}[fragile]
\frametitle{Modulo 7: Esempi Base di Sicurezza}
\textbf{Gestione degli Errori:} (già mostrato negli esempi di connessione al DB)
\vspace{1em}

\textbf{Sanitizzazione con \texttt{htmlspecialchars()}:} (integrato negli esempi di form)
\end{frame}

\begin{frame}[fragile]
\frametitle{Modulo 7: Esempi Aggiuntivi di Sicurezza}
\textbf{Utilizzo delle Sessioni per il Login:}
\begin{lstlisting}
<?php
  session_start();
  $_SESSION['utente'] = $username;
  echo "Login effettuato, benvenuto " . $_SESSION['utente'];
?>
\end{lstlisting}
\vspace{0.5em}
\textbf{Hashing delle Password:}
\begin{lstlisting}
<?php
  // Al momento dell'inserimento:
  $passwordHash = password_hash($password, PASSWORD_DEFAULT);
  // Salva $passwordHash nel database

  // Durante il login:
  if (password_verify($passwordInserita, $passwordHashDalDB)) {
      echo "Password corretta!";
  } else {
      echo "Password errata!";
  }
?>
\end{lstlisting}
\vspace{0.5em}
\textbf{Token CSRF in un Form:}
\begin{lstlisting}
<?php
  session_start();
  if (empty($_SESSION['token'])) {
      $_SESSION['token'] = bin2hex(random_bytes(32));
  }
?>
<!-- Nel form HTML -->
<input type="hidden" name="token" value="<?php echo $_SESSION['token']; ?>">
<?php
  if (hash_equals($_SESSION['token'], $_POST['token'])) {
      // Procedi con l'elaborazione
  } else {
      echo "Token non valido!";
  }
?>
\end{lstlisting}
\end{frame}

% ====================
% Conclusioni
% ====================

\begin{frame}
\frametitle{Conclusioni e Risorse Utili}
\begin{itemize}
    \item Riepilogo: Dalle basi di PHP all'interazione con database e sicurezza.
    \item Risorse:
    \begin{itemize}
        \item \href{https://www.php.net/manual/it/}{Documentazione ufficiale PHP}
        \item Forum e community online (StackOverflow, GitHub, ecc.)
    \end{itemize}
    \item Invito alla pratica e al lavoro di gruppo.
\end{itemize}
\end{frame}

\end{document}
